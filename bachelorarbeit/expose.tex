\documentclass[a4paper,12pt]{article}

% Pakete
\usepackage[german]{babel}
\usepackage[utf8]{inputenc}
\usepackage[T1]{fontenc}
\usepackage{lmodern}
\usepackage{setspace}
\usepackage{amsmath}
\usepackage{graphicx}
\usepackage{booktabs}
\usepackage{tabularx}
\usepackage{hyperref}
\usepackage[left=2.5cm,right=2.5cm,top=2.5cm,bottom=2.5cm]{geometry}

% Seitenränder und Zeilenabstand
\onehalfspacing

% Titel und Autor
\title{
    \vspace{2cm}
    \Huge\textbf{Exposé}
    \vspace{1.5cm}
    
    \Large 
    Vorhersage von menschlichen Mobilitätsmustern über mehrere Städte hinweg\\
    (GISCUP 2025)
}

\author{
    \vspace{2cm}
    \textbf{Vorgelegt von:} 230646 Ivan Pushkarev, \texttt{ivan.pushkarev@tu-dortmund.de}\\
    \textbf{Betreuer:} Mart Hagedoorn, M.Sc. \texttt{mart.hagedoorn@tu-dortmund.de}\\
    \textbf{Studiengang:} B. Sc. Informatik\\
    \textbf{Fakultät für Informatik, TU Dortmund}
}

\date{}

\begin{document}
\begin{titlepage}
\maketitle
\thispagestyle{empty}
\end{titlepage}

\section{Fragestellung}

Die Vorhersage menschlicher Mobilitätsmuster ist ein zentrales Forschungsgebiet mit vielfältigen Anwendungen in der Stadtplanung, im Transportwesen, im Katastrophenmanagement und in der Epidemiemodellierung. Durch die Verfügbarkeit großer Datensätze zur menschlichen Bewegung, die von Mobilgeräten und sozialen Medienplattformen gesammelt werden, konnten komplexe Modelle zur menschlichen Mobilitätsvorhersage entwickelt und getestet werden.

Der Human Mobility Prediction Challenge 2025 (GISCUP 2025) fokussiert sich auf die Vorhersage von Bewegungsmustern in vier japanischen Metropolregionen (Städte A, B, C und D) über einen Zeitraum von 75 Tagen.

Die zentrale Fragestellung des Wettbewerbs lautet: ``Können wir die Vorhersagbarkeit menschlicher Mobilität verbessern, indem wir Daten aus verschiedenen Städten nutzen?'' Diese Frage ist besonders relevant, da städteübergreifende Transferierbarkeit von Mobilitätsmodellen ein wichtiges und bisher unzureichend erforschtes Gebiet darstellt.

Bei dieser Aufgabe stehen verschiedene Herausforderungen im Vordergrund:

\begin{enumerate}
    \item Die Datensätze haben unterschiedliche Größen (Stadt A: 100.000 Personen, Stadt B: 25.000, Stadt C: 20.000, Stadt D: 6.000), was die Frage aufwirft, ob datenreiche Städte zur Verbesserung der Vorhersagen in datenarmen Städten beitragen können.
    
    \item Die räumliche Struktur jeder Stadt ist einzigartig, was die direkte Übertragbarkeit von Modellen erschwert.
    
    \item Für die Städte C und D liegen die vorherzusagenden Tage 61-75 zeitlich mehrere Monate nach den Tagen 1-60, was eine zusätzliche zeitliche Herausforderung darstellt.
\end{enumerate}

\textbf{Hauptfrage:}
\begin{itemize}
    \item Wie lässt sich die Vorhersagbarkeit menschlicher Mobilitätsmuster durch die Nutzung von Daten aus mehreren Städten verbessern?
\end{itemize}

\textbf{Teilfragen:}
\begin{itemize}
    \item Welche Mobilitätsmuster sind stadtspezifisch und welche sind stadtübergreifend beobachtbar?
    \item Wie kann Transfer Learning eingesetzt werden, um Wissen von datenreichen Städten auf datenarme Städte zu übertragen?
    \item Welche Methoden eignen sich besonders für die langfristige Vorhersage in Szenarien mit zeitlichen Lücken (wie bei Städten C und D)?
    \item Inwieweit verbessert die Integration von POI-Daten (Points of Interest) die Qualität der Vorhersage?
\end{itemize}

\section{Theoretischer Rahmen}

Die Arbeit stützt sich auf verschiedene theoretische Konzepte und Forschungsgebiete:

\subsection{Menschliche Mobilitätsmodelle}

Die Forschung zur menschlichen Mobilität hat verschiedene Modelle hervorgebracht, von einfachen Random Walks bis hin zu komplexen State-Space-Modellen. Grundlegende Arbeiten wie die von Brockmann et al. (2006) und González et al. (2008) haben gezeigt, dass menschliche Bewegungsmuster trotz ihrer scheinbaren Komplexität skalenfreien Lévy-Flügen mit charakteristischen Verteilungen folgen.

\subsection{Maschinelles Lernen für Zeitreihendaten}

Die Mobilitätsvorhersage lässt sich als Zeitreihenprognose darstellen. Moderne Deep-Learning-Ansätze wie Transformer-Architekturen (Vaswani et al., 2017) und große Sprachmodelle (LLMs) haben sich als vielversprechend für diese Aufgabe erwiesen, wie die Arbeiten von Tang et al. (2024) und He et al. (2024) zeigen.

\subsection{Transfer Learning und Domain Adaptation}

Zur Übertragung von Wissen zwischen verschiedenen Städten sind Techniken des Transfer Learnings und der Domain Adaptation relevant. Diese ermöglichen es, Modelle auf datenreichen Quellen zu trainieren und anschließend auf datenarme Zieldomänen anzupassen (Pan \& Yang, 2010).

\section{Methodische Herangehensweise}

\subsection{Retrieval-basierte Methode}

Als Hauptansatz wird eine retrieval-basierte Methode verfolgt. Dabei werden für einen neuen Tag und Nutzer ähnliche beobachtete Tage und Nutzerprofile aus dem Trainingsdatensatz identifiziert und deren Trajektorien als Basis für die Vorhersage genutzt. Dieser Ansatz hat den Vorteil, dass er weniger anfällig gegenüber exakten Verhaltensprognosen ist und stattdessen auf Ähnlichkeitsmustern basiert, die städteübergreifend gelten können.

Die Implementierung umfasst:
\begin{itemize}
    \item Entwicklung von Random-Walk-Modellen, die POI-Informationen als Gewichtungsfaktoren für Übergangswahrscheinlichkeiten zwischen Gitterzellen nutzen
    \item Definition geeigneter Ähnlichkeitsmetriken für den Vergleich von Trajektorien
    \item Implementierung effizienter Suchalgorithmen zur Identifikation ähnlicher Muster
\end{itemize}

\subsection{Evaluierung und Vergleich}

Die entwickelten Methoden werden anhand der vom GISCUP vorgegebenen Metriken evaluiert:
\begin{itemize}
    \item GEO-BLEU: Ein Ähnlichkeitsmaß für räumliche Sequenzen
    \item Dynamic Time Warping (DTW): Ein Maß für die Ähnlichkeit von Zeitreihen
\end{itemize}

Als Baseline-Modelle dienen die im Challenge-Dokument beschriebenen Ansätze (Global Mean, Global Mode, Per-User Mean, Per-User Mode, Unigram und Bigram Modelle).

\section{Daten und Materialien}

Die Arbeit verwendet den offiziellen GISCUP 2025 Datensatz, der menschliche Mobilitätsmuster in vier japanischen Städten enthält:
\begin{itemize}
    \item Stadt A: 100.000 Individuen
    \item Stadt B: 25.000 Individuen
    \item Stadt C: 20.000 Individuen
    \item Stadt D: 6.000 Individuen
\end{itemize}

Jede Stadt ist in ein 200×200 Raster unterteilt, wobei jede Zelle einem Gebiet von 500×500 Metern entspricht. Die Mobilitätsdaten decken einen Zeitraum von 75 Tagen ab und sind in 30-Minuten-Intervalle unterteilt.

Die Daten enthalten zusätzlich POI-Informationen (Points of Interest) für jede Gitterzelle als 85-dimensionalen Vektor, die als ergänzende Informationen genutzt werden können.

\section{Erwartete Ergebnisse}

Die Arbeit soll folgende Ergebnisse liefern:
\begin{enumerate}
    \item Eine detaillierte Analyse der stadtübergreifenden Gemeinsamkeiten und Unterschiede in Mobilitätsmustern
    \item Ein effektives Modell zur Vorhersage menschlicher Bewegung, das die GEO-BLEU- und DTW-Metriken der Baseline-Modelle übertrifft
    \item Erkenntnisse über die Transferierbarkeit von Mobilitätswissen zwischen verschiedenen urbanen Umgebungen
    \item Empfehlungen für zukünftige Ansätze im Bereich der stadtübergreifenden Mobilitätsvorhersage
\end{enumerate}

Die erwarteten Erkenntnisse sind nicht nur für den GISCUP 2025 relevant, sondern bieten auch wertvolle Einsichten für praktische Anwendungen in der Stadtplanung, im Katastrophenmanagement und in der Epidemiemodellierung.

\section{Zeitplan}

\begin{tabular}{p{2.5cm}|p{3.5cm}|p{8cm}}
\toprule
\textbf{Zeitraum} & \textbf{Phase} & \textbf{Aktivitäten} \\
\midrule
Woche 1-2 & Literaturrecherche & Einarbeitung in die Fachliteratur zur Mobilitätsvorhersage und Transfer Learning \\
\midrule
Woche 3-4 & Datenanalyse & Exploration und Visualisierung der Datensätze, Identifikation wichtiger Muster \\
\midrule
Woche 5-10 & Modellentwicklung & Implementierung der retrieval-basierten Methode \\
\midrule
Woche 11-12 & Evaluation & Vergleich der entwickelten Modelle mit Baselines anhand der GISCUP-Metriken \\
\midrule
Woche 13-14 & Dokumentation & Zusammenfassung der Ergebnisse, Abschluss der Bachelorarbeit \\
\bottomrule
\end{tabular}

\section{Literaturverzeichnis}

Brockmann, D., Hufnagel, L., \& Geisel, T. (2006). The scaling laws of human travel. \textit{Nature}, 439(7075), 462-465.

González, M. C., Hidalgo, C. A., \& Barabási, A. L. (2008). Understanding individual human mobility patterns.

Pan, S. J., \& Yang, Q. (2010). A survey on transfer learning. \textit{IEEE Transactions on knowledge and data engineering}, 22(10), 1345-1359.

Wikipedia. (2023). Matrix completion. Retrieved from \url{https://en.wikipedia.org/wiki/Matrix_completion}

Buchin, K., Hagedoorn, M., \& Korn, A. (2024). Discretized Random Walk Models for Efficient Movement Interpolation. In \textit{The 32nd ACM International Conference on Advances in Geographic Information Systems}.

He, H., Luo, H., \& Wang, Q. R. (2024). ST-MoE-BERT: A Spatial-Temporal Mixture-of-Experts Framework for Long-Term Cross-City Mobility Prediction. In \textit{2nd ACM SIGSPATIAL International Workshop on the Human Mobility Prediction Challenge}.

Tang, P., Yang, C., Xing, T., Xu, X., Jiang, R., \& Sezaki, K. (2024). Instruction-Tuning Llama-3-8B Excels in City-Scale Mobility Prediction. In \textit{2nd ACM SIGSPATIAL International Workshop on the Human Mobility Prediction Challenge}.

Terashima, H., Takagi, S., Tamura, N., Shoji, K., Hossain, T., Katayama, S., Urano, K., Yonezawa, T., \& Kawaguchi, N. (2024). Time-series Stay Frequency for Multi-City Next Location Prediction using Multiple BERTs. In \textit{2nd ACM SIGSPATIAL International Workshop on the Human Mobility Prediction Challenge}.

Vaswani, A., Shazeer, N., Parmar, N., Uszkoreit, J., Jones, L., Gomez, A. N., Kaiser, L., \& Polosukhin, I. (2017). Attention is all you need. \textit{Advances in neural information processing systems}, 30.

\end{document}